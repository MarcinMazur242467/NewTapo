 \section{Testowanie i Analiza wyników}
% \label{sec:testowanie}
% Wstęp: Ten rozdzial ma na celu:
% - przetestowanie  jak dziala rozwiazanie które zbudowalem 
% - wyciagniecie wnioskow z tego co mozna zrobic lepiej
% Oraz synterza z poprzednia cześcią pracy.  

% \subsection{Zakres Testów}
% \label{subsec:zakres_testow}
% Opis metodologii testowania funkcjonalności PTZ, streamingu, detekcji ruchu i nagrywania.
% Umożliwienie stabilnego \textbf{wyświetlania obrazu w czasie rzeczywistym} w medium. Weryfikacja nastąpi poprzez pomiar \textbf{opóźnienia strumienia wideo ($Latency$)} oraz wskaźnika \textbf{klatek na sekundę (FPS)}, celem osiągnięcia płynności monitoringu.
% Testy jednostkowe:
% - dlugosc nagrania ile zabiera ramu 
% - czy PTZ wykrwa blokade 
% - 

% - czy nagrania sie zapisuja
% Testy empiryczne:
% - Czy live preview dziala
% - Czy mozna oddtworzyc zapiasne nagranie 
% - czy dziala PTZ 
% - zmiana stanu na motion detected  
% Weryfikacja nastąpi poprzez analizę \textbf{efektywności algorytmów} mierzoną w kategoriach czasu przetwarzania klatki oraz minimalizacji błędów detekcji.

% \subsection{Środowisko Testowe}
% \label{subsec:srodowisko_testowe}
% Docker kamera tplink tapo c200 laptop 

% Specyfikacja sprzętu (np. Raspberry Pi lub PC hostujący Docker), wersji oprogramowania i konfiguracji sieciowej.

% \subsection{Wyniki testów i Analiza}
% \label{subsec:wyniki_i_analiza}
% Prezentacja kluczowych metryk, w tym:
% \begin{itemize}
%     \item \textbf{Opóźnienie (Latency):} Porównanie opóźnień strumienia RTSP przetworzonego przez Flask/WebSockets z oficjalną aplikacją.
%     \item \textbf{Skuteczność Detekcji Ruchu:} Wyniki testów algorytmu OpenCV (np. metryki True Positive Rate, False Positive Rate).
%     \item \textbf{Wydajność Systemu:} Obciążenie procesora hosta w warunkach ciągłego streamingu i detekcji.
% \end{itemize}

% \subsection{Podsumowanie}
% \label{subsec:podsumowanie_roz4}