 \section{Testowanie i Analiza wyników}
% \label{sec:testowanie}
% Wstęp: Ten rozdzial ma na celu:
% - przetestowanie  jak dziala rozwiazanie które zbudowalem 
% - wyciagniecie wnioskow z tego co mozna zrobic lepiej
% Oraz synterza z poprzednia cześcią pracy.  

\subsection{Zakres Testów}
\label{subsec:zakres_testow}
Opis metodologii testowania funkcjonalności PTZ, streamingu, detekcji ruchu i nagrywania.
Umożliwienie stabilnego wyświetlania obrazu w czasie rzeczywistym. Weryfikacja nastąpi poprzez pomiar opóźnienia strumienia wideo oraz wskaźnika klatek na sekundę, celem osiągnięcia płynności monitoringu.
Testy jednostkowe:
- dlugosc nagrania ile zabiera ramu 
- czy PTZ wykrwa blokade 
- 

% - czy nagrania sie zapisuja
% Testy empiryczne:
% - Czy live preview dziala
% - Czy mozna oddtworzyc zapiasne nagranie 
% - czy dziala PTZ 
% - zmiana stanu na motion detected  
% Weryfikacja nastąpi poprzez analizę \textbf{efektywności algorytmów} mierzoną w kategoriach czasu przetwarzania klatki oraz minimalizacji błędów detekcji.

% \subsection{Środowisko Testowe}
% \label{subsec:srodowisko_testowe}
% Docker kamera tplink tapo c200 laptop 

% Specyfikacja sprzętu (np. Raspberry Pi lub PC hostujący Docker), wersji oprogramowania i konfiguracji sieciowej.

% \subsection{Wyniki testów i Analiza}
% \label{subsec:wyniki_i_analiza}
% Prezentacja kluczowych metryk, w tym:
% \begin{itemize}
%     \item \textbf{Opóźnienie (Latency):} Porównanie opóźnień strumienia RTSP przetworzonego przez Flask/WebSockets z oficjalną aplikacją.
%     \item \textbf{Skuteczność Detekcji Ruchu:} Wyniki testów algorytmu OpenCV (np. metryki True Positive Rate, False Positive Rate).
%     \item \textbf{Wydajność Systemu:} Obciążenie procesora hosta w warunkach ciągłego streamingu i detekcji.
% \end{itemize}
\subsection{Wyniki testów i Analiza}
\label{subsec:wyniki_i_analiza}

W celu kwantyfikacji wydajności oraz stabilności zaimplementowanego systemu, przeprowadzono serię testów, których zagregowane wyniki przedstawiono we wcześniejszej tabeli. Analiza objęła trzy fundamentalne aspekty: wydajność algorytmu detekcji ruchu, rzeczywistą płynność i stabilność strumienia wideo, oraz zachowanie systemu w kontekście zarządzania pamięcią.

\subsubsection*{Analiza wydajności detekcji ruchu (Test T01)}
Test wydajności komponentu analitycznego wykazał, że średni czas potrzebny na przetworzenie klatki 1080p przez algorytm detekcji ruchu wynosi \textbf{65.1 ms}. Implikuje to teoretyczną, maksymalną przepustowość na poziomie około \textbf{15.36 FPS}, identyfikując ten proces jako potencjalne ,,wąskie gardło'' w scenariuszach wymagających analizy z wysoką płynnością.

\subsubsection*{Analiza rzeczywistej wydajności i stabilności streamingu (Test T02, T02.1, T02.2)}
W celu uzyskania wiarygodnych danych, test wydajności streamingu na żywo przeprowadzono wielokrotnie, zbierając łącznie \textbf{347 próbek} w docelowym środowisku kontenerowym. Uśredniona wydajność systemu wyniosła \textbf{15.70 FPS}, co zapewnia płynne doświadczenie dla użytkownika końcowego. Co ważniejsze, obliczone odchylenie standardowe na poziomie zaledwie \textbf{1.57 FPS} ilościowo potwierdza wysoką ogólną stabilność strumienia.
Jednakże, podczas testów zaobserwowano również istotne zdarzenia rzadkie. Wystąpiły sporadyczne spadki wydajności do zaledwie \textbf{10 FPS}, które były skorelowane z pojawianiem się w logach systemowych błędów dekodera H.264 (np. \texttt{error while decoding MB}) oraz nieobsłużonych wyjątków prowadzących do zamknięcia procesu (\texttt{terminate called without an active exception}). Obserwacje te sugerują, że chociaż system jest generalnie stabilny, mogą w nim występować chwilowe problemy z integralnością danych w strumieniu RTSP, co prowadzi do błędów w niższej warstwie przetwarzania wideo. Jest to kluczowy wniosek wskazujący na potencjalne obszary dalszego rozwoju, takie jak implementacja mechanizmów odporności na błędy (fault tolerance).

\subsubsection*{Analiza zużycia pamięci RAM (Test T03)}
Test symulujący proces nagrywania wideo ujawnił krytyczną cechę architektury modułu zapisu. Zaobserwowano liniowy przyrost zużycia pamięci operacyjnej w tempie około \textbf{148.3 MB na sekundę}. Wynik ten jednoznacznie potwierdza, że mechanizm buforowania klatek w pamięci RAM jest nieskalowalny. Wnioskiem inżynierskim jest tu konieczność refaktoryzacji modułu zapisu w celu implementacji mechanizmu zapisu strumieniowego bezpośrednio na dysk (ang. \textit{stream-to-disk}), co pozwoli na realizację długotrwałych nagrań przy zachowaniu stałego, niskiego zużycia zasobów.

\begin{figure}[h!]
    \centering
    \includegraphics[width=0.9\textwidth]{Sources/wykres_stabilnosci_fps_aktualizacja.png}
    \caption{Wykres stabilności FPS z N=347 próbek (Test T02).}
    \label{fig:fps_stability}
\end{figure}

\begin{figure}[h!]
    \centering
    \includegraphics[width=0.9\textwidth]{Sources/wykres_zuzycia_ram.png}
    \caption{Wykres liniowego wzrostu zużycia pamięci RAM (Test T03).}
    \label{fig:ram_usage}
\end{figure}


% \subsection{Podsumowanie}
% \label{subsec:podsumowanie_roz4}