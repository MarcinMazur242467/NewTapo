\newpage
\phantomsection
\section*{Wstęp}
\addcontentsline{toc}{section}{Wstęp}

Globalny rynek systemów monitoringu przechodzi dynamiczną transformację, będącą efektem rozwoju \textbf{Internetu Rzeczy}. Kamery IP stały się wszechobecnym elementem infrastruktury cyfrowej, pełniąc funkcje od podstawowego dozoru, aż po zaawansowaną analizę danych. Równolegle z postępem technologicznym, pojawia się wyzwanie o charakterze inżynierskim, jakim jest występowanie systemów opartych na \textbf{zamkniętym oprogramowaniu}.\\

Wybór tematu pracy wynika z konieczności zaadresowania problemu \textbf{ograniczonego potencjału sprzętowego} w kontekście popularnej serii kamer konsumenckich TP-Link Tapo. Zjawisko to, polegające na uzależnieniu pełnej funkcjonalności sprzętu od infrastruktury chmurowej i aplikacji mobilnej producenta, ogranicza dostepność danych na poszczególnych platformach i stanowi ograniczenia nagrywania i późniejszego wykorzystania sprzętu do lokalnych zastosowań.\\

W pracy zastosowano \textbf{metodykę Double Diamond}, dzieląc proces projektowy na fazy eksploracji i definiowania problemu (analiza protokołów kamery) oraz fazy rozwoju i dostarczania rozwiązania. Warstwa aplikacyjna została zaimplementowana w języku \textbf{Python 3.13} z wykorzystaniem \textbf{konteneryzacji Docker} dla zapewnienia izolacji i wysokiej \textbf{reprodukowalności środowiska}. Komunikacja z kamerą odbywa się poprzez bibliotekę \textbf{PyTapo}, natomiast przetwarzanie strumienia wideo RTSP realizują narzędzia \textbf{PyAV} i \textbf{OpenCV}. Taki zestaw narzędzi, osadzony w architekturze serwera \textbf{Flask} z protokołem \textbf{WebSocket's}, pozwolił na stworzenie systemu o niskim opóźnieniu i wysokiej niezawodności.\\