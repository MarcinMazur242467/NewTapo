\newpage
\phantomsection
\section*{Streszczenie}
\addcontentsline{toc}{section}{Streszczenie}

Niniejsza praca inżynierska podejmuje problematykę ograniczonego potencjału sprzętowego w konsumenckich kamerach IoT, na przykładzie modelu TP-Link Tapo C200.

Głównym celem pracy było zaprojektowanie i implementacja autorskiego, modułowego oprogramowania typu Open Source, pozwalającego na obsługę kamery w sieci lokalnej, z pominięciem dedykowanej aplikacji producenta. Projekt zrealizowano zgodnie z metodyką Double Diamond, a rozwiązanie wdrożono w języku Python 3.13 z wykorzystaniem konteneryzacji Docker.

W warstwie technologicznej zastosowano bibliotekę OpenCV do przetwarzania obrazu i autorskiej detekcji ruchu, PyAV do obsługi ścieżki audio oraz bibliotekę PyTapo, która implementuje sterowanie mechaniką PTZ (Pan-Tilt-Zoom) za pomocą inżynierii wstecznej protokołów producenta. Interfejs użytkownika zbudowano w oparciu o serwer Flask i protokół WebSocket, co pozwoliło na komunikację i podgląd w czasie rzeczywistym.

Przeprowadzone testy wydajnościowe potwierdziły osiągnięcie stabilnego strumieniowania wideo oraz skuteczność algorytmu detekcji ruchu działającego na brzegu sieci. Zidentyfikowano również ograniczenia związane z narzutem środowiska Python oraz zarządzaniem pamięcią RAM podczas rejestracji nagrań. 

Wnioski, które płyną z pracy inżynierskiej, pozwoliły na sformułowanie rekomendacji dotyczących dalszego rozwoju niniejszego rozwiązania. Poprzez zastąpienie języka Python szybszym językiem programowania w celu implementacji kluczowych modułów operujących na analizie obrazu, zamianę algorytmu odroczonego zapisu na rzecz algorytmu stosującego zapis strumieniowy na dysk urządzenia oraz implementacji mechanizmu persystancji danych, prototyp aplikacji osiągnąłby swoją pełną funkcjonalność i byłby gotowy do wdrożenia.

\vspace{1cm}
\section*{Słowa kluczowe}
\addcontentsline{toc}{subsection}{Słowa kluczowe}
IoT, Kamera IP, TP-Link Tapo, Open Source, PyTapo, Docker, RTSP, Detekcja Ruchu, Flask.